% Options for packages loaded elsewhere
\PassOptionsToPackage{unicode}{hyperref}
\PassOptionsToPackage{hyphens}{url}
%
\documentclass[
]{book}
\usepackage{amsmath,amssymb}
\usepackage{lmodern}
\usepackage{iftex}
\ifPDFTeX
  \usepackage[T1]{fontenc}
  \usepackage[utf8]{inputenc}
  \usepackage{textcomp} % provide euro and other symbols
\else % if luatex or xetex
  \usepackage{unicode-math}
  \defaultfontfeatures{Scale=MatchLowercase}
  \defaultfontfeatures[\rmfamily]{Ligatures=TeX,Scale=1}
\fi
% Use upquote if available, for straight quotes in verbatim environments
\IfFileExists{upquote.sty}{\usepackage{upquote}}{}
\IfFileExists{microtype.sty}{% use microtype if available
  \usepackage[]{microtype}
  \UseMicrotypeSet[protrusion]{basicmath} % disable protrusion for tt fonts
}{}
\makeatletter
\@ifundefined{KOMAClassName}{% if non-KOMA class
  \IfFileExists{parskip.sty}{%
    \usepackage{parskip}
  }{% else
    \setlength{\parindent}{0pt}
    \setlength{\parskip}{6pt plus 2pt minus 1pt}}
}{% if KOMA class
  \KOMAoptions{parskip=half}}
\makeatother
\usepackage{xcolor}
\usepackage{color}
\usepackage{fancyvrb}
\newcommand{\VerbBar}{|}
\newcommand{\VERB}{\Verb[commandchars=\\\{\}]}
\DefineVerbatimEnvironment{Highlighting}{Verbatim}{commandchars=\\\{\}}
% Add ',fontsize=\small' for more characters per line
\usepackage{framed}
\definecolor{shadecolor}{RGB}{248,248,248}
\newenvironment{Shaded}{\begin{snugshade}}{\end{snugshade}}
\newcommand{\AlertTok}[1]{\textcolor[rgb]{0.94,0.16,0.16}{#1}}
\newcommand{\AnnotationTok}[1]{\textcolor[rgb]{0.56,0.35,0.01}{\textbf{\textit{#1}}}}
\newcommand{\AttributeTok}[1]{\textcolor[rgb]{0.77,0.63,0.00}{#1}}
\newcommand{\BaseNTok}[1]{\textcolor[rgb]{0.00,0.00,0.81}{#1}}
\newcommand{\BuiltInTok}[1]{#1}
\newcommand{\CharTok}[1]{\textcolor[rgb]{0.31,0.60,0.02}{#1}}
\newcommand{\CommentTok}[1]{\textcolor[rgb]{0.56,0.35,0.01}{\textit{#1}}}
\newcommand{\CommentVarTok}[1]{\textcolor[rgb]{0.56,0.35,0.01}{\textbf{\textit{#1}}}}
\newcommand{\ConstantTok}[1]{\textcolor[rgb]{0.00,0.00,0.00}{#1}}
\newcommand{\ControlFlowTok}[1]{\textcolor[rgb]{0.13,0.29,0.53}{\textbf{#1}}}
\newcommand{\DataTypeTok}[1]{\textcolor[rgb]{0.13,0.29,0.53}{#1}}
\newcommand{\DecValTok}[1]{\textcolor[rgb]{0.00,0.00,0.81}{#1}}
\newcommand{\DocumentationTok}[1]{\textcolor[rgb]{0.56,0.35,0.01}{\textbf{\textit{#1}}}}
\newcommand{\ErrorTok}[1]{\textcolor[rgb]{0.64,0.00,0.00}{\textbf{#1}}}
\newcommand{\ExtensionTok}[1]{#1}
\newcommand{\FloatTok}[1]{\textcolor[rgb]{0.00,0.00,0.81}{#1}}
\newcommand{\FunctionTok}[1]{\textcolor[rgb]{0.00,0.00,0.00}{#1}}
\newcommand{\ImportTok}[1]{#1}
\newcommand{\InformationTok}[1]{\textcolor[rgb]{0.56,0.35,0.01}{\textbf{\textit{#1}}}}
\newcommand{\KeywordTok}[1]{\textcolor[rgb]{0.13,0.29,0.53}{\textbf{#1}}}
\newcommand{\NormalTok}[1]{#1}
\newcommand{\OperatorTok}[1]{\textcolor[rgb]{0.81,0.36,0.00}{\textbf{#1}}}
\newcommand{\OtherTok}[1]{\textcolor[rgb]{0.56,0.35,0.01}{#1}}
\newcommand{\PreprocessorTok}[1]{\textcolor[rgb]{0.56,0.35,0.01}{\textit{#1}}}
\newcommand{\RegionMarkerTok}[1]{#1}
\newcommand{\SpecialCharTok}[1]{\textcolor[rgb]{0.00,0.00,0.00}{#1}}
\newcommand{\SpecialStringTok}[1]{\textcolor[rgb]{0.31,0.60,0.02}{#1}}
\newcommand{\StringTok}[1]{\textcolor[rgb]{0.31,0.60,0.02}{#1}}
\newcommand{\VariableTok}[1]{\textcolor[rgb]{0.00,0.00,0.00}{#1}}
\newcommand{\VerbatimStringTok}[1]{\textcolor[rgb]{0.31,0.60,0.02}{#1}}
\newcommand{\WarningTok}[1]{\textcolor[rgb]{0.56,0.35,0.01}{\textbf{\textit{#1}}}}
\usepackage{longtable,booktabs,array}
\usepackage{calc} % for calculating minipage widths
% Correct order of tables after \paragraph or \subparagraph
\usepackage{etoolbox}
\makeatletter
\patchcmd\longtable{\par}{\if@noskipsec\mbox{}\fi\par}{}{}
\makeatother
% Allow footnotes in longtable head/foot
\IfFileExists{footnotehyper.sty}{\usepackage{footnotehyper}}{\usepackage{footnote}}
\makesavenoteenv{longtable}
\usepackage{graphicx}
\makeatletter
\def\maxwidth{\ifdim\Gin@nat@width>\linewidth\linewidth\else\Gin@nat@width\fi}
\def\maxheight{\ifdim\Gin@nat@height>\textheight\textheight\else\Gin@nat@height\fi}
\makeatother
% Scale images if necessary, so that they will not overflow the page
% margins by default, and it is still possible to overwrite the defaults
% using explicit options in \includegraphics[width, height, ...]{}
\setkeys{Gin}{width=\maxwidth,height=\maxheight,keepaspectratio}
% Set default figure placement to htbp
\makeatletter
\def\fps@figure{htbp}
\makeatother
\setlength{\emergencystretch}{3em} % prevent overfull lines
\providecommand{\tightlist}{%
  \setlength{\itemsep}{0pt}\setlength{\parskip}{0pt}}
\setcounter{secnumdepth}{5}
\usepackage{booktabs}
\ifLuaTeX
  \usepackage{selnolig}  % disable illegal ligatures
\fi
\usepackage[]{natbib}
\bibliographystyle{plainnat}
\IfFileExists{bookmark.sty}{\usepackage{bookmark}}{\usepackage{hyperref}}
\IfFileExists{xurl.sty}{\usepackage{xurl}}{} % add URL line breaks if available
\urlstyle{same} % disable monospaced font for URLs
\hypersetup{
  pdftitle={Introdução à linguagem R},
  pdfauthor={Jean S. S. Resende, Jéssica M. Magno, João C. D. Muzzi, Mauro A. A. Castro},
  hidelinks,
  pdfcreator={LaTeX via pandoc}}

\title{Introdução à linguagem R}
\author{Jean S. S. Resende, Jéssica M. Magno, João C. D. Muzzi, Mauro A. A. Castro}
\date{2023-06-04}

\usepackage{amsthm}
\newtheorem{theorem}{Theorem}[chapter]
\newtheorem{lemma}{Lemma}[chapter]
\newtheorem{corollary}{Corollary}[chapter]
\newtheorem{proposition}{Proposition}[chapter]
\newtheorem{conjecture}{Conjecture}[chapter]
\theoremstyle{definition}
\newtheorem{definition}{Definition}[chapter]
\theoremstyle{definition}
\newtheorem{example}{Example}[chapter]
\theoremstyle{definition}
\newtheorem{exercise}{Exercise}[chapter]
\theoremstyle{definition}
\newtheorem{hypothesis}{Hypothesis}[chapter]
\theoremstyle{remark}
\newtheorem*{remark}{Remark}
\newtheorem*{solution}{Solution}
\begin{document}
\maketitle

{
\setcounter{tocdepth}{1}
\tableofcontents
}
\hypertarget{prefuxe1cio}{%
\chapter{Prefácio}\label{prefuxe1cio}}

A linguagem R foi criada por professores do departamento de estatística da univer sidade de Auckland no ano 2000. A intenção destes professores era disponibilizar uma linguagem \emph{open-source} para computação estatística. Com o avanço e popularização da linguagem, profissionais de diversas áreas passaram a utilizá-la em suas análises de dados.

Esta apostila contém uma introdução à linguagem de programação R. Aborda desde conteúdos teóricos quanto práticos, com exemplos didáticos a fim de facilitar o aprendizado. Esta apostila foi produzida principalmente pelos autores: Jean Silva de Souza Resende, Jéssica Maria Magno, João Carlos Degram Muzzi sob orientação de Mauro Antônio Alves Castro. Em versões anteriores, tivemos a colaboração dos autores: Sheyla Trefflich, Danrley R. Fernandes e Giuseppe Pasqualato Neto.

\hypertarget{introduuxe7uxe3o}{%
\chapter{Introdução}\label{introduuxe7uxe3o}}

\hypertarget{contextualizando-a-linguagem-de-programauxe7uxe3o-r}{%
\section{Contextualizando a linguagem de programação R}\label{contextualizando-a-linguagem-de-programauxe7uxe3o-r}}

A linguagem R é uma linguagem de programação com o foco em computação estatística e manipulação de gráficos. Criada no início dos anos 90 por \href{https://en.wikipedia.org/wiki/Ross_Ihaka}{Geroge Ross Ihaka} e \href{https://en.wikipedia.org/wiki/Robert_Gentleman_(statistician)}{Robert Clifford Gentleman}, o R é usado mais utilizado por estatísticos, bioinformatas, analistas de dados e desenvolvedor de \emph{software} estatístico. No entanto ele tem se destacado na comunidade científica. Em maio de 2023, o R ocupava a 16ª posição no \href{https://www.tiobe.com/tiobe-index/}{índice TIOBE}, uma medida de popularidade da linguagem de programação, sendo que em agosto de 2020 o R atingiu seu pico em ficando 8º lugar.

O R é um ambiente de \emph{software} livre de código aberto, disponível sob a \href{https://en.wikipedia.org/wiki/GNU_General_Public_License}{\emph{GNU General Public License}}. Seus executáveis pré-compilados são fornecidos para vários sistemas operacionais. Ele tem uma interface de linha de comando, mas também possui interfaces gráficas de usuário (GUI) de terceiros como o \href{https://posit.co/download/rstudio-desktop/}{Rstudio} - que será a IDE (\emph{Integrated Development Envirenment}) que iremos utilizar na apostila.

\hypertarget{instalauxe7uxe3o-do-r}{%
\section{Instalação do R}\label{instalauxe7uxe3o-do-r}}

\begin{enumerate}
\def\labelenumi{\arabic{enumi}.}
\tightlist
\item
  Acesse o repositório do R (\href{https://cran-r.c3sl.ufpr.br/}{clique aqui}).
\item
  Acesse o link referente ao seu sitema operacional: Linux, macOS ou Windows.
  2.1. Linux: escolha a distribuição linux (debian, fedora, redhat, suse ou ubuntu) e então prossiga com os comandos no terminal.
  2.2. macOS: escolha o instalador conforme o modelo da sua máquina e execute-o.
  2.3. Windows: acesse o link Base e então baixe o instalador e execute-o.
\end{enumerate}

\hypertarget{instalauxe7uxe3o-do-rstudio}{%
\section{Instalação do RStudio}\label{instalauxe7uxe3o-do-rstudio}}

\begin{enumerate}
\def\labelenumi{\arabic{enumi}.}
\tightlist
\item
  \href{https://posit.co/download/rstudio-desktop/}{Clique aqui} para acessar o repositório do RStudio.
\item
  Baixe o instalador conforme o sistema operacional da sua máquina (Linux/macOS/Windows).
\end{enumerate}

\hypertarget{interface-do-rstudio}{%
\section{Interface do RStudio}\label{interface-do-rstudio}}

Por padrão o RStudio abre quatro janelas (pode ocorrer de uma estar oculta, mas observe o botão de minimizar/maximizar no canto superior direito de cada janela).

\begin{itemize}
\tightlist
\item
  \textbf{Editor de código} (\emph{janela do canto superior esquerdo}): Nela você digita os comando a serem executados no RStudio. Para executá-los aperte as teclas `CTRL' e `ENTER' simultaneamente na linha ou bloco de código selecionado.
\item
  \textbf{Console} (\emph{janela do canto inferior esquerdo}): É visto as saídas dos comandos que são rodados. Também é possível digitar e rodar códigos diretamente nesta janela.
\item
  \textbf{Histórico} (\emph{janela do canto superior direito}): Nesta janela ficam salvos os objetos, históricos de comandos e conexões com outros aplicativos.
\item
  \textbf{Visualização} (\emph{janela do canto inferior direito}): Aqui você pode visualizar os gráficos no RStudio, navegar entre os arquivos do seu computador, visualizar os pacotes instalados e ver a ajuda de comandos e descrições de tabelas de dados e por fim navegar entre os arquivos html.
\end{itemize}

\begin{figure}
\centering
\includegraphics{figure/fig1_rstudio.png}
\caption{Janelas do RStudio. Ref: \url{https://www.est.ufmg.br/~cristianocs/Pacotes2021/Intro.html\#6}}
\end{figure}

\hypertarget{pacotes}{%
\section{Pacotes}\label{pacotes}}

Por ser \emph{Open Source}, o R permite que qualquer usuário disponibilize funções e bancos de dados a comunidade. As funções/bancos de dados são disponibilizados através de pacotes. A instalação de um pacote depende do repositório que ele está armazenado: máquina local, CRAN, GitHub, Bioconductor, entre outros. O repostitório CRAN contêm muitos pacotes e não é direcionado à uma área específica (como é o caso do Bioconductor que se destina a pacotes voltados para área de biotecnologia). A instalação de um pacote do repositório CRAN é feita pelo menu ``Tools \textgreater{} Install Packages'' ou simplesmente utilizando o seguinte comando:

\begin{Shaded}
\begin{Highlighting}[]
\FunctionTok{installed.packages}\NormalTok{(}\StringTok{"nomeDoPacote"}\NormalTok{)}
\end{Highlighting}
\end{Shaded}

Para que você possa utilizar as funções do pacote que instalou, você deve usar um dos dois comandos a seguir, para de fato carregar as funções do pacote para o ambiente R:

\begin{Shaded}
\begin{Highlighting}[]
\FunctionTok{library}\NormalTok{(nomeDoPacote)}
\FunctionTok{require}\NormalTok{(nomeDoPacote)}
\end{Highlighting}
\end{Shaded}

A função \texttt{library()} é utilizada normalmente no corpo do script, enquanto que a função \texttt{require()} é utilizada dentro de outras funções.

\hypertarget{comeuxe7ando-de-fato-a-programar-em-r}{%
\section{Começando de fato a programar em R}\label{comeuxe7ando-de-fato-a-programar-em-r}}

No console (janela do canto inferior esquerdo) digite o comando a seguir e tecle ENTER:

\begin{Shaded}
\begin{Highlighting}[]
\FunctionTok{print}\NormalTok{(}\StringTok{"Hello World"}\NormalTok{)}
\end{Highlighting}
\end{Shaded}

Agora digite o mesmo comando no editor de código (janela do canto superior esquerdo) e com o cursor na mesma linha do comando, tecle CTRL e ENTER simultaneamente:

\begin{Shaded}
\begin{Highlighting}[]
\FunctionTok{print}\NormalTok{(}\StringTok{"Hello World"}\NormalTok{)}
\end{Highlighting}
\end{Shaded}

A diferença é que quando executamos os comandos no editor de código, o comando continua no editor para ser executado, ou seja svocê está construindo um script. Mas você executa comando diretamente no console, eles não ficam gravados em um editor.

O dado de saída da função, foi um print do que estava dentro da função. Mas como você saber o que usar dentro de uma determinada função, como \texttt{print()}? Você precisa acessar o manual desta função.

\hypertarget{acessando-o-manual-da-funuxe7uxe3o}{%
\section{Acessando o manual da função}\label{acessando-o-manual-da-funuxe7uxe3o}}

Esta é uma etapa muito importante que antecede a sua caminhada no aprendizado do R. Você pode visualizar o manual da função executando um comando onde um ponto de interregoção (?) precisa anteceder a função:

\begin{Shaded}
\begin{Highlighting}[]
\NormalTok{?print}
\end{Highlighting}
\end{Shaded}

Mas se você deseja encontrar funções que realizam uma determinada ação, basta inserir dois pontos de interrogação antecedendo a ação desejada:

\begin{Shaded}
\begin{Highlighting}[]
\NormalTok{??priting}
\end{Highlighting}
\end{Shaded}

O comando acima realizará uma busca por tópicos que contenham a palavra \emph{ploting}. Outra opçõe alternativa ao \textbf{?} é o uso da função \texttt{help()} e \texttt{help.search()} para \textbf{??}

\begin{Shaded}
\begin{Highlighting}[]
\FunctionTok{help}\NormalTok{(}\StringTok{"print"}\NormalTok{)}
\FunctionTok{help.search}\NormalTok{(}\StringTok{"priting"}\NormalTok{)}
\end{Highlighting}
\end{Shaded}

Algumas funções possuem exemplos de sua execução. Se você quer saber como utilizar uma determinada função através de exemplos, execute a função \texttt{example()}.

\begin{Shaded}
\begin{Highlighting}[]
\FunctionTok{example}\NormalTok{(}\StringTok{"print"}\NormalTok{)}
\end{Highlighting}
\end{Shaded}

\hypertarget{comentando-cuxf3digos-no-r}{%
\section{Comentando códigos no R}\label{comentando-cuxf3digos-no-r}}

A maioria das linguagens de programação e até linguagem de marcação, possuem uma forma de inserção de textos que não serão executados pela linguagem. Esse procedimento é denominado de comentário. Você pode comentar os seus códigos. Isso é algo essencial para todos os programadores, indiferente da linguagem. Pois, códigos comentados facilitam a interpretação do mesmo por outros programadores e até mesmo pelo autor, devido a um período de tempo que se passou desde a criação daquele código.

Para comentar linhas no R você precisa inserir o \texttt{\#} antes do que seria o comentário:

\begin{Shaded}
\begin{Highlighting}[]
\CommentTok{\# isto é um comentário}
\end{Highlighting}
\end{Shaded}

Exemplo aplicado:

\begin{Shaded}
\begin{Highlighting}[]
\FunctionTok{print}\NormalTok{(}\StringTok{"Hello World"}\NormalTok{) }\CommentTok{\# imprimindo na tela Hello World}
\end{Highlighting}
\end{Shaded}

Perceba que o conteúdo após o \texttt{\#} não é interpretado no R, ou seja, este conteúdo é um comentário.

\hypertarget{fundamentos-buxe1sicos-da-programauxe7uxe3o}{%
\chapter{Fundamentos básicos da programação}\label{fundamentos-buxe1sicos-da-programauxe7uxe3o}}

\hypertarget{variuxe1veis}{%
\section{Variáveis}\label{variuxe1veis}}

Utilizamos a variável para armazenar um valor qualquer em um local da memória RAM do computador. Deste modo, é possível reutilizar esse valor, usando o nome da sua variável.

\hypertarget{declarauxe7uxe3o-e-atribuiuxe7uxe3o-de-variuxe1veis}{%
\subsection{Declaração e atribuição de variáveis}\label{declarauxe7uxe3o-e-atribuiuxe7uxe3o-de-variuxe1veis}}

Em R declaramos uma variável atribuindo a ela um valor em três formas diferente: \textbf{símbolo de atribuição \textless-}, \textbf{símbolo de atribuição =} e \textbf{função assign()}.

\begin{Shaded}
\begin{Highlighting}[]
\NormalTok{nome.var }\OtherTok{\textless{}{-}}\NormalTok{ valor }\CommentTok{\# atribuicao: menor e traco}
\NormalTok{nome.var }\OtherTok{=}\NormalTok{ valor }\CommentTok{\# atribuicao: igual}
\FunctionTok{assign}\NormalTok{(}\StringTok{"nome\_var"}\NormalTok{,valor) }\CommentTok{\# funcao: assign}
\end{Highlighting}
\end{Shaded}

\hypertarget{dicas-para-nomear-variuxe1veis}{%
\subsection{Dicas para nomear variáveis}\label{dicas-para-nomear-variuxe1veis}}

As variáveis podem ser nomeadas com o uso letras, números, ponto (.) e underline (\_), no entanto é necessário se atentar para algumas dicas de como nomear as variáveis:

\begin{enumerate}
\def\labelenumi{\arabic{enumi}.}
\item
  O nome da variável deve sempre começar com uma letra ou um ponto, ou seja, não pode iniciar com números ou símbolos. Se iniciar com ponto o próximo caracter não pode ser um número.
\item
  O nome da variável que contêm mais de uma palavra é recomendado o uso do underline (\_) para separa-la.
\item
  O nome da variável não pode ser palavras reservadas da linguagem como TRUE, if, while, entre outras.
\item
  O nome da variável não pode conter espaços.
\item
  O nome da variável deve ser condizente com o seu valor.
\end{enumerate}

\hypertarget{tipos-de-dados-das-variuxe1veis}{%
\subsection{Tipos de dados das variáveis}\label{tipos-de-dados-das-variuxe1veis}}

Em R o tipo de dado da variável é obtido a partir do valor atribuído à ela. Isto faz da linguagem R: \textbf{Linguagem dinamicamente tipada}. Pois, o tipo de dado de uma variável pode ser alterado dinamicamente enquanto o programa/script é executado.

As variáveis em R podem ser do tipo: inteiro (interger), ponto flutuante (double), complexo (complex), caracteres (character/string) e lógico (logical).

\begin{Shaded}
\begin{Highlighting}[]
\NormalTok{var\_int }\OtherTok{\textless{}{-}}\NormalTok{ 2L      }\CommentTok{\# var integer}
\NormalTok{var\_db1 }\OtherTok{\textless{}{-}} \FloatTok{1.5}     \CommentTok{\# var double}
\NormalTok{var\_db2 }\OtherTok{\textless{}{-}} \DecValTok{2}       \CommentTok{\# var double}
\NormalTok{var\_comp }\OtherTok{\textless{}{-}} \DecValTok{2} \SpecialCharTok{+}\NormalTok{ 3i }\CommentTok{\# var complex}
\NormalTok{var\_str }\OtherTok{\textless{}{-}} \StringTok{"a\_01"}  \CommentTok{\# var string/character}
\NormalTok{var\_log }\OtherTok{\textless{}{-}} \ConstantTok{TRUE}    \CommentTok{\# var logical}
\end{Highlighting}
\end{Shaded}

Podemos verificar o tipo das variáveis criadas no \emph{chunk} anterior através da função \texttt{typeof()}.

\begin{Shaded}
\begin{Highlighting}[]
\FunctionTok{typeof}\NormalTok{(var\_int)}
\end{Highlighting}
\end{Shaded}

\begin{verbatim}
## [1] "integer"
\end{verbatim}

\begin{Shaded}
\begin{Highlighting}[]
\FunctionTok{typeof}\NormalTok{(var\_db1)}
\end{Highlighting}
\end{Shaded}

\begin{verbatim}
## [1] "double"
\end{verbatim}

\begin{Shaded}
\begin{Highlighting}[]
\FunctionTok{typeof}\NormalTok{(var\_db2)}
\end{Highlighting}
\end{Shaded}

\begin{verbatim}
## [1] "double"
\end{verbatim}

\begin{Shaded}
\begin{Highlighting}[]
\FunctionTok{typeof}\NormalTok{(var\_comp)}
\end{Highlighting}
\end{Shaded}

\begin{verbatim}
## [1] "complex"
\end{verbatim}

\begin{Shaded}
\begin{Highlighting}[]
\FunctionTok{typeof}\NormalTok{(var\_str)}
\end{Highlighting}
\end{Shaded}

\begin{verbatim}
## [1] "character"
\end{verbatim}

\begin{Shaded}
\begin{Highlighting}[]
\FunctionTok{typeof}\NormalTok{(var\_log)}
\end{Highlighting}
\end{Shaded}

\begin{verbatim}
## [1] "logical"
\end{verbatim}

Para verificar quais variáveis o R está usando \emph{workspace} usando a função \texttt{ls()}.

\begin{Shaded}
\begin{Highlighting}[]
\FunctionTok{ls}\NormalTok{()}
\end{Highlighting}
\end{Shaded}

\begin{verbatim}
## [1] "var_comp" "var_db1"  "var_db2"  "var_int"  "var_log"  "var_str"
\end{verbatim}

Para excluir variáveis, ou seja, desalocar determinada variável da memória RAM, basta usar a função \texttt{rm()}.

\begin{Shaded}
\begin{Highlighting}[]
\FunctionTok{rm}\NormalTok{(var\_str)     }\CommentTok{\# desaloca a variavel var\_str}
\FunctionTok{rm}\NormalTok{(}\AttributeTok{list =} \FunctionTok{ls}\NormalTok{()) }\CommentTok{\# desaloca todas as variaveis}
\end{Highlighting}
\end{Shaded}

\hypertarget{operauxe7uxf5es-em-r}{%
\section{Operações em R}\label{operauxe7uxf5es-em-r}}

Podemos executar operações matemáticas, lógicas e comparações em R. Para isso o R faz uso de \textbf{operadores}. Os operadores são divididos em: aritmético, relacional e lógico.

Os operadores aritméticos como o nome já diz são usados em operações aritméticas e são eles:

\begin{itemize}
\tightlist
\item
  Adição: +
\item
  Subtração: -
\item
  Multiplicação: *
\item
  Divisão: /
\item
  Resto de divisão: \%\%
\item
  Divisão inteira: \%/\%
\item
  Potenciação: \^{}
\end{itemize}

\begin{Shaded}
\begin{Highlighting}[]
\DecValTok{2}\SpecialCharTok{+}\DecValTok{2}   \CommentTok{\# soma}
\DecValTok{5{-}2}   \CommentTok{\# subtracao}
\DecValTok{2}\SpecialCharTok{*}\DecValTok{5}   \CommentTok{\# multiplicacao}
\DecValTok{5}\SpecialCharTok{/}\DecValTok{2}   \CommentTok{\# divisao}
\DecValTok{5}\SpecialCharTok{\%\%}\DecValTok{2}  \CommentTok{\# resto de divisao}
\DecValTok{5}\SpecialCharTok{\%/\%}\DecValTok{2} \CommentTok{\# divisao inteira}
\DecValTok{2}\SpecialCharTok{\^{}}\DecValTok{5}   \CommentTok{\# potenciacao}
\end{Highlighting}
\end{Shaded}

Já os operadores relacionais, tratam da relação de um valor com o outro e são eles:

\begin{itemize}
\tightlist
\item
  Menor: \textless{}
\item
  Maior: \textgreater{}
\item
  Menor ou igual: \textless=
\item
  Maior ou igual: \textgreater=
\item
  Igual: ==
\item
  Diferente: !=
\end{itemize}

\begin{Shaded}
\begin{Highlighting}[]
\DecValTok{2}\SpecialCharTok{\textless{}}\DecValTok{5} \CommentTok{\# menor}
\DecValTok{2}\SpecialCharTok{\textgreater{}}\DecValTok{5} \CommentTok{\# maior}
\DecValTok{2}\SpecialCharTok{\textless{}=}\DecValTok{2} \CommentTok{\# menor ou igual}
\DecValTok{2}\SpecialCharTok{\textgreater{}=}\DecValTok{5} \CommentTok{\# maior ou igual}
\DecValTok{5}\SpecialCharTok{==}\DecValTok{5} \CommentTok{\# igual}
\DecValTok{2}\SpecialCharTok{!=}\DecValTok{2} \CommentTok{\# diferente}
\end{Highlighting}
\end{Shaded}

Por fim, os operadores lógicos são:

\begin{itemize}
\tightlist
\item
  \emph{logical NOT}: !
\item
  \emph{logical AND}: \&
\item
  \emph{logical OR}: \textbar{}
\end{itemize}

\begin{Shaded}
\begin{Highlighting}[]
\SpecialCharTok{!}\ConstantTok{TRUE} \CommentTok{\# NOT = qual e o contrario de TRUE?}
\ConstantTok{TRUE} \SpecialCharTok{|} \ConstantTok{FALSE} \CommentTok{\# OR = um dos dois ou os dois é ou são verdadeiros?}
\ConstantTok{TRUE} \SpecialCharTok{\&} \ConstantTok{FALSE} \CommentTok{\# AND = os dois são verdadeiros?}
\end{Highlighting}
\end{Shaded}

\hypertarget{condiuxe7uxf5es-e-loops}{%
\section{Condições e loops}\label{condiuxe7uxf5es-e-loops}}

Existem dois passos que são trilhados por toda linguagem de programação, e alguns progrmadores dizem que se uma linguagem de programação não permite a execução destes dois passos, ela não é bem considerada uma linguagem de programação. Um exemplo é a linguagem HTML, essa linguagem é dita como \textbf{linguagem de marcação} sua finalidade é trabalhar com estruturação de textos. Não iremos utilizá-la para cálculos ou procedimentos que demandam de uma rotina computacional com base em cálculos e nos dois passos. Mas quais são estes dois passos? R: condições e \emph{loops}.

\hypertarget{condiuxe7uxf5es}{%
\subsection{Condições}\label{condiuxe7uxf5es}}

Se alguma coisa for verdadeira (TRUE) o R vai agir de uma maneira, caso seja mentira (FALSE) ele vai agir de outra maneira. Você pode estabelecer algumas condições para que seja feita uma função.

\hypertarget{condiuxe7uxe3o-if-if}{%
\subsubsection{\texorpdfstring{Condição: if \texttt{if()}}{Condição: if if()}}\label{condiuxe7uxe3o-if-if}}

Determinado código será executado somente se a condição for verdadeira, abaixo é apresentado a estrura do \texttt{if}.

\begin{Shaded}
\begin{Highlighting}[]
\CommentTok{\# {-}{-} estrutura}

\CommentTok{\# if(condicao)\{}
\CommentTok{\#   comandos a serem executados}
\CommentTok{\# \}}
\end{Highlighting}
\end{Shaded}

Vamos agora fazer uma aplicação: se o número dois for maior que o número um, então imprima na tela a frase: dois é maior que um. Caso contrário não faça nada.

\begin{Shaded}
\begin{Highlighting}[]
\CommentTok{\# {-}{-} aplicacao}
\DocumentationTok{\#\# {-}{-} verdadeiro}
\ControlFlowTok{if}\NormalTok{(}\DecValTok{2}\SpecialCharTok{\textgreater{}}\DecValTok{1}\NormalTok{)\{}
  \FunctionTok{print}\NormalTok{(}\StringTok{"dois é maior que um"}\NormalTok{)}
\NormalTok{\}}
\end{Highlighting}
\end{Shaded}

\begin{verbatim}
## [1] "dois é maior que um"
\end{verbatim}

No exemplo abaixo a condição é falsa, logo o comando dentro de if não é executado.

\begin{Shaded}
\begin{Highlighting}[]
\DocumentationTok{\#\# {-}{-} falso}
\ControlFlowTok{if}\NormalTok{(}\DecValTok{2}\SpecialCharTok{\textless{}}\DecValTok{1}\NormalTok{)\{}
  \FunctionTok{print}\NormalTok{(}\StringTok{"dois é menor que um"}\NormalTok{)}
\NormalTok{\}}
\end{Highlighting}
\end{Shaded}

\hypertarget{condiuxe7uxe3o-if-else-if-else}{%
\subsubsection{\texorpdfstring{Condição: if else \texttt{if()\ else()}}{Condição: if else if() else()}}\label{condiuxe7uxe3o-if-else-if-else}}

Podemos querer que um comando seja executado se condição for verdadeira e outro comando seja executado se a condição for falsa. Faremos da seguinte forma:

\begin{Shaded}
\begin{Highlighting}[]
\ControlFlowTok{if}\NormalTok{(}\ConstantTok{TRUE}\NormalTok{)\{}
  \FunctionTok{print}\NormalTok{(}\StringTok{"comando dentro do if"}\NormalTok{)}
\NormalTok{\}}\ControlFlowTok{else}\NormalTok{\{}
  \FunctionTok{print}\NormalTok{(}\StringTok{"comando dentro do else"}\NormalTok{)}
\NormalTok{\}}
\end{Highlighting}
\end{Shaded}

\begin{verbatim}
## [1] "comando dentro do if"
\end{verbatim}

Por ser verdadeira a condição dentro do \texttt{if}, foi executado o primeiro comando.

\begin{Shaded}
\begin{Highlighting}[]
\ControlFlowTok{if}\NormalTok{(}\ConstantTok{FALSE}\NormalTok{)\{}
  \FunctionTok{print}\NormalTok{(}\StringTok{"comando dentro do if"}\NormalTok{)}
\NormalTok{\}}\ControlFlowTok{else}\NormalTok{\{}
  \FunctionTok{print}\NormalTok{(}\StringTok{"comando dentro do else"}\NormalTok{)}
\NormalTok{\}}
\end{Highlighting}
\end{Shaded}

\begin{verbatim}
## [1] "comando dentro do else"
\end{verbatim}

A condição dentro do \texttt{if} é falsa então foi executado o comando dentro do \texttt{else}

Outra forma de aplicar a condição if else é usando a função \texttt{ifelse()}.

\begin{Shaded}
\begin{Highlighting}[]
\FunctionTok{ifelse}\NormalTok{(}\DecValTok{2} \SpecialCharTok{\textgreater{}} \DecValTok{1}\NormalTok{, }\DecValTok{2}\SpecialCharTok{*}\DecValTok{1}\NormalTok{, }\DecValTok{1}\SpecialCharTok{/}\DecValTok{2}\NormalTok{) }\CommentTok{\# condicao verdadeira}
\end{Highlighting}
\end{Shaded}

\begin{verbatim}
## [1] 2
\end{verbatim}

\begin{Shaded}
\begin{Highlighting}[]
\FunctionTok{ifelse}\NormalTok{(}\DecValTok{2} \SpecialCharTok{\textless{}} \DecValTok{1}\NormalTok{, }\DecValTok{2}\SpecialCharTok{*}\DecValTok{1}\NormalTok{, }\DecValTok{1}\SpecialCharTok{/}\DecValTok{2}\NormalTok{) }\CommentTok{\# condicao falsa}
\end{Highlighting}
\end{Shaded}

\begin{verbatim}
## [1] 0.5
\end{verbatim}

\hypertarget{loops}{%
\subsection{\texorpdfstring{\emph{Loops}}{Loops}}\label{loops}}

É muito trabalhoso reescrever código a fim de obeter repetições, sem mencionar o tempo gasto nesta reescrita. Sendo assi, o R possui algumas funções de repetições são elas: \texttt{for()}, \texttt{while()} e \texttt{repeat()}.

A função \texttt{for()} repete o código para o comprimento da sequência indicada à ela.

\begin{Shaded}
\begin{Highlighting}[]
\ControlFlowTok{for}\NormalTok{(variavel }\ControlFlowTok{in}\NormalTok{ sequencia)\{}
\NormalTok{  comandos a serem repetidos}
\NormalTok{\}}
\end{Highlighting}
\end{Shaded}

No exemplo abaixo a variável \emph{i} vai assumir um valor da sequência numérica 1, 2, 3, 4 e 5, e então executar a função \texttt{print()} em i para cada valor da sequência adotada por \emph{i}.

\begin{Shaded}
\begin{Highlighting}[]
\CommentTok{\# : cria uma sequencia Ex.: sequencia do 1 ao 5 = 1:5}
\ControlFlowTok{for}\NormalTok{(i }\ControlFlowTok{in} \DecValTok{1}\SpecialCharTok{:}\DecValTok{5}\NormalTok{)\{ }
  \FunctionTok{print}\NormalTok{(i)}
\NormalTok{\}}
\end{Highlighting}
\end{Shaded}

\begin{verbatim}
## [1] 1
## [1] 2
## [1] 3
## [1] 4
## [1] 5
\end{verbatim}

Outro exemplo do uso do for: Vamos printar na tela as cindo primeiras letras do alfabeto.

\begin{Shaded}
\begin{Highlighting}[]
\ControlFlowTok{for}\NormalTok{(letra }\ControlFlowTok{in}\NormalTok{ letters[}\DecValTok{1}\SpecialCharTok{:}\DecValTok{5}\NormalTok{])\{}
  \FunctionTok{print}\NormalTok{(letra)}
\NormalTok{\}}
\end{Highlighting}
\end{Shaded}

\begin{verbatim}
## [1] "a"
## [1] "b"
## [1] "c"
## [1] "d"
## [1] "e"
\end{verbatim}

Já a função \texttt{while()} executa os comandos enquanto a condição informada a ela for verdadeira.

\begin{Shaded}
\begin{Highlighting}[]
\ControlFlowTok{while}\NormalTok{(condição)\{}
\NormalTok{  comandos a serem repetidos}
\NormalTok{\}}
\end{Highlighting}
\end{Shaded}

Por exempĺo: vamos construir um temporizador que determina um espaço de tempo de cinco segundos.

\begin{Shaded}
\begin{Highlighting}[]
\NormalTok{contador }\OtherTok{\textless{}{-}} \DecValTok{1}
\ControlFlowTok{while}\NormalTok{(contador }\SpecialCharTok{\textless{}=} \DecValTok{5}\NormalTok{)\{}
  \FunctionTok{print}\NormalTok{(contador)}
\NormalTok{  contador }\OtherTok{=}\NormalTok{ contador }\SpecialCharTok{+} \DecValTok{1}
\NormalTok{\}}
\end{Highlighting}
\end{Shaded}

\begin{verbatim}
## [1] 1
## [1] 2
## [1] 3
## [1] 4
## [1] 5
\end{verbatim}

Perceba que não foi exatamente um espaço de tempo de cinco segundos, foi mais rápido. Vamos inserir um comando ao R dizendo a ele para aguardar um segundo após a execução anterior.

\begin{Shaded}
\begin{Highlighting}[]
\NormalTok{contador }\OtherTok{\textless{}{-}} \DecValTok{1}
\ControlFlowTok{while}\NormalTok{(contador }\SpecialCharTok{\textless{}=} \DecValTok{5}\NormalTok{)\{}
  \FunctionTok{print}\NormalTok{(contador)}
\NormalTok{  contador }\OtherTok{=}\NormalTok{ contador }\SpecialCharTok{+} \DecValTok{1}
  \FunctionTok{Sys.sleep}\NormalTok{(}\DecValTok{1}\NormalTok{)}
\NormalTok{\}}
\end{Highlighting}
\end{Shaded}

\begin{verbatim}
## [1] 1
## [1] 2
## [1] 3
## [1] 4
## [1] 5
\end{verbatim}

A função \texttt{repeat()} é usada quando queremos repetir um código sem a avaliação de uma condição. Atenção: vamos precisar utilizazr a função \texttt{break()} para dizer ao programa o momento em que deve parar a execução, ou seja a repetição. Também utilizaremos a função \texttt{if()} para avaliar a condição e então chamar o \textbf{break}.

\begin{Shaded}
\begin{Highlighting}[]
\NormalTok{contador }\OtherTok{\textless{}{-}} \DecValTok{10}
\ControlFlowTok{repeat}\NormalTok{\{}
  \FunctionTok{print}\NormalTok{(contador)}
\NormalTok{  contador }\OtherTok{\textless{}{-}}\NormalTok{ contador }\SpecialCharTok{+} \DecValTok{10}
  \ControlFlowTok{if}\NormalTok{(contador }\SpecialCharTok{\textgreater{}} \DecValTok{100}\NormalTok{) }\ControlFlowTok{break}\NormalTok{()}
\NormalTok{\}}
\end{Highlighting}
\end{Shaded}

\begin{verbatim}
## [1] 10
## [1] 20
## [1] 30
## [1] 40
## [1] 50
## [1] 60
## [1] 70
## [1] 80
## [1] 90
## [1] 100
\end{verbatim}

\hypertarget{lista-de-exercuxedcios}{%
\section{Lista de exercícios}\label{lista-de-exercuxedcios}}

\begin{enumerate}
\def\labelenumi{\arabic{enumi}.}
\item
  Declare três variáveis atribuindo valores numéricos e apresente o resultado da multiplicação das suas combinações dois a dois destas três variáveis (cada variável com um número). Ex.: variáveis A, B, e C mostre AxB, AxC e BxC com atribuição dos valores as variáveis.
\item
  Converta (no R) a temperatura Fahrenheit 78 °F para Centígrados. Fórmula: C = (F-32) x (5/9).
\item
  Calcule (no R):
\end{enumerate}

\begin{itemize}
\tightlist
\item
  o resto da divisão de 7 por 9
\item
  2 elevado ao cubo
\item
  raiz quadrada de 64
\end{itemize}

\begin{enumerate}
\def\labelenumi{\arabic{enumi}.}
\setcounter{enumi}{3}
\tightlist
\item
  Elabore um algoritmo que:
\end{enumerate}

\begin{itemize}
\tightlist
\item
  crie um vetor com uma sequência numérica de 5 números
\item
  faça um loop para calcular a soma destes números
\end{itemize}

\begin{enumerate}
\def\labelenumi{\arabic{enumi}.}
\setcounter{enumi}{4}
\tightlist
\item
  Elabore um algoritmo que:
\end{enumerate}

\begin{itemize}
\tightlist
\item
  crie um vetor com uma sequência numérica de 7 números
\item
  faça um loop para calcular a média destes números
\end{itemize}

\begin{enumerate}
\def\labelenumi{\arabic{enumi}.}
\setcounter{enumi}{5}
\tightlist
\item
  Elabore um algoritmo que:
\end{enumerate}

\begin{itemize}
\tightlist
\item
  crie um vetor com uma sequência numérica de 12 números
\item
  faça a soma dos números pares
\end{itemize}

\hypertarget{fundamentos-do-r}{%
\chapter{Fundamentos do R}\label{fundamentos-do-r}}

\hypertarget{processamento-de-dados}{%
\chapter{Processamento de dados}\label{processamento-de-dados}}

\hypertarget{blocks}{%
\chapter{Blocks}\label{blocks}}

\hypertarget{equations}{%
\section{Equations}\label{equations}}

Here is an equation.

\begin{equation} 
  f\left(k\right) = \binom{n}{k} p^k\left(1-p\right)^{n-k}
  \label{eq:binom}
\end{equation}

You may refer to using \texttt{\textbackslash{}@ref(eq:binom)}, like see Equation \eqref{eq:binom}.

\hypertarget{theorems-and-proofs}{%
\section{Theorems and proofs}\label{theorems-and-proofs}}

Labeled theorems can be referenced in text using \texttt{\textbackslash{}@ref(thm:tri)}, for example, check out this smart theorem \ref{thm:tri}.

\begin{theorem}
\protect\hypertarget{thm:tri}{}\label{thm:tri}For a right triangle, if \(c\) denotes the \emph{length} of the hypotenuse
and \(a\) and \(b\) denote the lengths of the \textbf{other} two sides, we have
\[a^2 + b^2 = c^2\]
\end{theorem}

Read more here \url{https://bookdown.org/yihui/bookdown/markdown-extensions-by-bookdown.html}.

\hypertarget{callout-blocks}{%
\section{Callout blocks}\label{callout-blocks}}

The R Markdown Cookbook provides more help on how to use custom blocks to design your own callouts: \url{https://bookdown.org/yihui/rmarkdown-cookbook/custom-blocks.html}

\hypertarget{sharing-your-book}{%
\chapter{Sharing your book}\label{sharing-your-book}}

\hypertarget{publishing}{%
\section{Publishing}\label{publishing}}

HTML books can be published online, see: \url{https://bookdown.org/yihui/bookdown/publishing.html}

\hypertarget{pages}{%
\section{404 pages}\label{pages}}

By default, users will be directed to a 404 page if they try to access a webpage that cannot be found. If you'd like to customize your 404 page instead of using the default, you may add either a \texttt{\_404.Rmd} or \texttt{\_404.md} file to your project root and use code and/or Markdown syntax.

\hypertarget{metadata-for-sharing}{%
\section{Metadata for sharing}\label{metadata-for-sharing}}

Bookdown HTML books will provide HTML metadata for social sharing on platforms like Twitter, Facebook, and LinkedIn, using information you provide in the \texttt{index.Rmd} YAML. To setup, set the \texttt{url} for your book and the path to your \texttt{cover-image} file. Your book's \texttt{title} and \texttt{description} are also used.

This \texttt{gitbook} uses the same social sharing data across all chapters in your book- all links shared will look the same.

Specify your book's source repository on GitHub using the \texttt{edit} key under the configuration options in the \texttt{\_output.yml} file, which allows users to suggest an edit by linking to a chapter's source file.

Read more about the features of this output format here:

\url{https://pkgs.rstudio.com/bookdown/reference/gitbook.html}

Or use:

\begin{Shaded}
\begin{Highlighting}[]
\NormalTok{?bookdown}\SpecialCharTok{::}\NormalTok{gitbook}
\end{Highlighting}
\end{Shaded}


  \bibliography{book.bib,packages.bib}

\end{document}
